% Chapter Template

\chapter{Pulsars and Neutron stars} % Main chapter title

\label{Chapter2} % Change X to a consecutive number; for referencing this chapter elsewhere, use \ref{ChapterX}

\lhead{Chapter 2. \emph{Pulsars and Neutron stars}} % Change X to a consecutive number; this is for the header on each page - perhaps a shortened title

%----------------------------------------------------------------------------------------
%    SECTION 1
%----------------------------------------------------------------------------------------
\section{Discovery}
\label{section1.1}
A pulsar is a highly magnetize and rotating neutron star that emits electromagnetic radiations in the form of a beam of emission. The first discovery of the pulsar was in July 1968 by Jocelyn Bell (\citet{hewish1968observation}), while she was investigating the angular structure of the compact radio sources by observing the scintillation caused by the irregular structure of the interplanetary medium (\citet{hewish1964detection}). This carried out by using the Mullard Radio Astronomy Observatory which operating at a frequency of 81 MHz. Four months later in November, the systematic investigation and analysis show a very high stable signal of pulses with a period of 1.337 s. The nature of this signal was first considered to be a man-made signal from the space or terrestrial signal from the moon but non of them have accepted. however, they state that:\\
"\textit{A tentative explanation of this unusual source in terms of the stable oscillation of white dwarf or neutron star is proposed}".\\
In 1934, there was a propose from two astronomers, Walter Baade and Fritz Zwicky who suggested a new form of the star as the result of the supernova explosion of a massive star (\citet{baade1934remarks}). The neutron star has extreme physical conditions compare to a white dwarf;  thus,  many studies (e.g \citet{oppenheimer1939massive}) have supported this model which used the equation of state for cold Fermi gas to show that a quasi-static solution is required to interpret the collapse of a large mass in small and dense core, which ultimately enable them to predict the density and the total mass. In 1967, Pacini also suggested that a high dense stellar core with strong magnetic field could be left behind the supernova explosion and as the result, it could be the source of the energy in the Crab Nebula (\citet{pacini1967energy}) also see Figure (\ref{fig:crab}). These studies show strong evidence that the new discovery of pulsar is a spinning neutron star.

\begin{figure}[htbp!] 
\centering    
\includegraphics[width=1.0\textwidth]{psr.JPG}
\caption[First detection of pulsar]{Discovery observations of the first pulsar. (a) The first recording of PSR 1919+21; the signal resembled the radio interference also seen on this chart. (b) Fast chart recording showing individual pulses as downward deflections of the trace. From \citet{hewish1964detection}}
\label{fig:psr}
\end{figure}

\begin{figure}[H] 
\centering    
\includegraphics[width=1.0\textwidth]{crab_nebula_pulsar_lo.jpg}
\caption[Crub Pulsar]{The Crab Pulsar and the Crab Nebula are one impressive pair, especially in this unique image made by a duo of NRAO radio telescopes. The Nebula was formed when the original star exploded: this 'supernova' explosion was so bright that in 1054 A.D it was visible in the daytime for several weeks. In the centuries that followed the remnant kept expanding. Then, in 1968, another product of the supernova was found, an object that turned out to be the engine powering the bright remnant: the Crab Pulsar. As the outer layers of the original star were ejected in the supernova, the entire core must have collapsed to form a pulsing neutron star or 'pulsar', one of the densest objects we know of in the entire galaxy.
The image courtesy of NRAO/AUI and Joeri van Leeuwen (UC Berkeley)/ESO/ AURA (\url{http://images.nrao.edu/592}).}
\label{fig:crab}
\end{figure}


\section{Pulsar and Neutron stars properties}
%Add introduction

\subsection{Basic properties}
% Add contents here
\textbf{mass}: The properties of the neutron stars can be deduced by using the equation of stat (EoS),(e.g \citet{oppenheimer1939massive}). Based on EoS model, the maximum mass of the neutron star predicted to be about 2 $M_\odot{}[\footnote{
The minimum and maximum values of the neutron star's masses are vary in the literatures, e.g \citet{lyne2012pulsar} which shows the possible maximum value of the neutron star mass is 3 $M_\odot{}$}]$ (see \citet{lattimer2001neutron}; \citet{lorimer2005handbook}).\\
\textbf{radius and density}: With regard to the neutron star diameters, the mass of many binary systems measured with high accuracy to be $ \approx $ 1.35 $M_\odot{}$. and therefore, the range of EoS for such mass support an equivalent radius in the ranges of 10.5 and 11.2 km. With a similar conditions, the density of the neutron star is $\rho = 6.7 \times 10^{14}$ g $cm^{-3}$ ($>$ density of the nuclear matter $\rho_s = 2.7 \times 10^{14} cm^{-3} $).\\
Consequently, the upper limit of the radius of the neutron star can be deduced by taking into account the stability of the neutron star due to the centrifugal force. For neutron star with mass M and radius R, rotating with angular velocity $\Omega$, for a period $P = 2 \pi/\Omega$ the radius can be given in the form:\\

\begin{equation}
\label{radius}
R = 1.5 \times 10^{3} \left( \frac{M}{M_\odot{}} \right)^{1/3} P^{2/3}km.
\end{equation}

For the fastest known pulsar, PSR J1748-2446ad with the period $P = 1.40$ ms ($\sim 716$ Hz or rotation per second)( 
\citet{hessels2006radio}); consider the mass $m = 1.35 M\odot{}$, this gives a radius of upper limit $R = 21.5$km. \citet{lyne2012pulsar}\\

\textbf{spin-down luminosity}: The lost of rotational kinetic energy is considered to be the reason of the observed spin down that seen on the received pulses, which increases the period of the pulsar as $\dot{P} = \frac{dp}{dt}$. The lost of energy can be given by

\begin{equation}
\label{spin}
\dot E \equiv- \frac{dE_{rot}}{dt} = -I\Omega\dot{\Omega}.
\end{equation}

where $\Omega = 2\pi/P$ is the rotational angular frequency and $I$ is the moment of inertia of the neutron star ($I = 10^{45}$ g $cm^2$), this equation shows the total emitted power from the neutron star.\\

\textbf{spin down and characteristic ages}: The spin down model can be further expressed in terms of rotational frequency ($\nu = 1/P$) by

\begin{equation}
\label{spin down}
\dot \nu = -K \nu^n.
\end{equation}
where K is constant and n is the so-called "braking index" (shows the spin-down behavior of the star and for simplest case, n $\sim 3$).\\
Rewriting equation (\ref{spin down}) in terms of pulse period, $\frac{dp}{dt} \dot{P} = K P^{2-n}$, then integrate and assuming K and n $\neq 1$. therefore, the age of the pulsar can be approximated as

\begin{equation}
\label{age}
T=\frac{P}{(n-1) \dot{P}} \left[1- \left(\frac{P_0}{P} \right)^{n-1}  \right]
\end{equation}
where $P_0 $ represents the spin period at the birth of the star. We can simplify equation (\ref{age}) by assuming $P_{0} << P$ and the spin-down is caused by the  magnetic dipole radiation, n=3 

\begin{equation}
\tau = \frac{P}{2\dot{P}}
\end{equation}

\textbf{Magnetic field}\\
The magnetic moment can be related to the magnetic field strength as $B \approx \frac{m}{r^3}$. then, assuming the neutron star with radius $R=10$ km and the moment of inertia $I = 19^{45} $ g $cm^2$; therefore, we can determine the magnetic field of the pulsar in terms of their period and period derivatives as the following,

\begin{equation}
B = 3.2 \times 10^{19}G \sqrt[]{P \dot{P} }.
\end{equation}
 

\subsection{Rotating dipole model}
The pulsar is a rotating Neutron star with a very strong magnetic field which emits radiation in the different electromagnetic spectrum. Although the emission mechanism of the pulsars has not fully understood yet, many models have been proposed to explain their mechanism. The first model was made by \citet{goldreich1969pulsar}. In this model, the neutron star assumed to have a dense magnetosphere; however, the particles within the region will be electrostatically accelerated by the strong magnetic field lines. The particles then gain energy, due to the acceleration, to escape through the open field lines. see Figure (\ref{fig:dipole model1})

\begin{figure}[H] 
\centering    
\includegraphics[width=0.8\textwidth]{JLmodel.png}
\caption[The Goldreich-Julian Model
Goldreich]{Illustrates how the process of the emission mechanis
m from the polar gab with the electon-positron cascade.
Figure from Hand book pulsar astronomy by \citet{lorimer2005handbook} }
\label{fig:dipole model1}
\end{figure}

Following the first model (\citet{goldreich1969pulsar}), the second attempt was carried by \citet{sturrock1971model} when he proposed a new mechanism of the "polar caps" (the area where the open field lines reached the light cylinder and connect with the surface of the star). however, each polar cape has electrons and protons polar zone. The electrons then accelerated along the open magnetic-field lines which leads to produce a $\gamma$ ray emission due to the curvature radiation. If the pulsar with a short period, $> 1$ second, the pairs of the electron-positron will be constructed and accelerated, for the second time, to produce more emission in the form of pair cascade. Furthermore, \citet{ruderman1975theory} also proposed a new model related to the previous one (e.g \citet{sturrock1971model}). The new model is the so-called the "polar gap model" which suggested that the open field lines are extended to high altitude from the stellar surface by the polar magnetosphere gap. This will create a difference between the top and the base of the gap, as the result, the gap "spark" by generating electron-positron pairs which allow the emitted emission to be observed. see Figure (\ref{fig:dipole model})

\begin{figure}[H] 
\centering    
\includegraphics[width=0.8\textwidth]{PSRs_pulsar_sketch.png}
\caption[magnetosphere model]{Shows the rotating dipole of the pulsar emission.
Figure from Hand book pulsar astronomy by \citet{lorimer2005handbook} }
\label{fig:dipole model}
\end{figure}


\subsection{Galactic distribution}
\label{GD}
The standard model of neutron stars formation proposed that the neutron stars are the result of supernova explosions of massive stars (\citet{lyne1994high}). More than 2600 pulsars are now known (\footnote{This is ATNF Pulsar Catalogue, version:1.57 \url{http://www.atnf.csiro.au/research/pulsar/psrcat/}}). Figure (\ref{fig:GD}) indicates that most of the pulsars are located near the center of the galactic plane which supports the standard model of the birth of neutron stars. In 1970 \citet{gunn1970nature} formulated a statistical method to study the magnetic-dipole model of pulsar's observations which leads to the first suggestion about the high-velocity of the pulsars using the existence of both, the proper motion in the Crab Nebula and the O-B stars as anomalously massive stars category as evidence to support this idea. By analyzing the distribution of pulsars perpendicular to the galactic plane, the authors found that the pulsars are likely born with a velocity of $100^{-1}s$ which agree with the previous dynamic hypothesis. This implies a "kick" during the birth depending on the edge of the object and similarly, it explains why the young pulsars are close to plan and the old luminous pulsars with the very long period appear far away from the galactic plane with an isotropic distribution. (see the open circle in Figure (\ref{fig:GD}) ), this known as Millisecond Pulsars. Subsection (\Cref{MSP}). 

\begin{figure}[ht!] 
\centering    
\includegraphics[width=1.0\textwidth]{GD.png}
\caption[Galactic Distribution of pulsars]{Shows the Hammer-Aitoff projection of the sky in Galactic coordinates. The filled circles show the Pulsar-supernova and the open circles show the millisecond pulsars. Image from Handbook of Pulsar Astronomy
book }
\label{fig:GD}
\end{figure}

\section{Pulsar categories and their evolution}
%NOTE to add the ref for different section 
\subsection{ \texorpdfstring{$P-\dot{P}$}{} Diagram}
The observed emission from pulsars, which is a result of the rotational kinetic energy of the neutron star, enables us to measure the pulsar's spin period, P, and the correspond spin-down rate, $\dot{P}$, of the pulsar with very high precisions measurements. Using a smiler idea of Hertzsprung- Russell diagram to study the evolution of the stars by plotting the luminosities versus the stellar classifications, the $P-\dot{P}$ diagram gives us an excellent overview of the spin evolution and other properties of the neutron stars by plotting the period and it's derivative (P and $\dot{P}$ respectively). Figure (\ref{fig:pp}) shows different types of pulsars, In the center we see the normal pulsar population (\cref{Normal Pulsars}) with the black dot, they are young and slow. Upper right we see the highly magnetized neutron stars called Magnetars with the green triangle. Button left show the binary system of pulsar which the origin of Millisecond Pulsars. We see also different types of pulsars associated with supernova remnants present by blue squares. Finally, we can note the yellow squares which show the Rotating Radio Transients.

\begin{figure}[H] 
\centering    
\includegraphics[width=1.0\textwidth]{sallycooper.png}
\caption[P--$\dot{P}$ Diagram]{: P--$\dot{P}$ diagram for known pulsar until 2016, the black dot in the middle right present the Normal pulsars "island", blue squares shows pulsars associated with supernova remnant (SNR), the purple circles in lower left shows the millisecond pulsars, the yellow poxes in the center to the right show the Rotating Radio Transients (RRATs) where the green triangle in the top right show the highly magnetize type of pulsars called Magnetars. Image made by \url{Sally_ Cooper}}
\label{fig:pp}
\end{figure}


\subsection{Normal pulsars}
\label{Normal Pulsars}
From the diagram in Figure (\ref{fig:pp}) it's clear that normal pulsars have surface magnetic field strength an order of $~10^{11}$ to $~10^{13}$ G, spin period of 0.1 to 1.0 second and period derivatives in order of $~10^{-16}$ to $~10^{-14}$ s $s^{-1}$. The normal pulsars have short spin period when they form (at the upper left-hand in Figure (\ref{fig:pp})), then followed by spin down and as a result, they move  toward the center (pulsar island) to have a characteristic ages $~10^{5}$ to $~10^{8}$ Years in their evolutionary tracks until finally, they become very faint to be detected after ~$10^{8}$ yr. More information about the characteristic of the normal pulsar can be found in (\url{https://arxiv.org/abs/astro-ph/0208557v1}).  

\subsection{Millisecond pulsars}
\label{MSP}
The second class of the pulsar populations with very short period and high spin frequency rate is so-called Millisecond Pulsars. As we look at Figure (\ref{fig:pp}), we can clearly spot the unique location of the millisecond pulsars population in the left button of P--$\dot{P}$ diagram with the magnetic field $10^{8}$ G and characteristic ages $10^{7}$  yr. One of the mean properties of millisecond pulsars is that they have orbiting companions observed at around 80\% of the total number of millisecond pulsars compare to 1\% for the normal pulsars. These orbiting companions could be one of the following, mean sequence stars, white dwarfs or neutron stars.\\
Following a similar evolutionary track of the normal pulsars discussed in (\cref{GD}), when the pulsar in the binary system last over $~10^{7}$ yr, it becomes faint (the energy of the star decreases and thus insufficient radio emission will be released), and undergo spin down which will become very slow with spin period in order of several seconds only. The model of the binary system formations has proposed to describe the formation of millisecond pulsar (\citet{alpar1982new}).\\
The formation and evolution model for different binaries has shown in figure (\ref{fig:mspf}), as the pulsar in the binary system spin down, some binary systems remain bounded with their companions. The massive companion then expands and evolve to become a giant star, as a result, the distance between the pulsar and its companion will decrease which enables the companion to fill it's Roche Lobe and accrete the matter (and subsequently transfers the orbital momentum) into the old neutron star to create an accretion disk. the pulsar start spinning up again for short periods, this is also called "recycled pulsar". in this stage the X-rays emission will be released and the system is called X-ray binary.\\
Depending on the mass of the companion, two classes can be identified, high-mass X-ray binaries ($HMXB_{s}$) and low-mass X-ray binaries ($LMXB_{s}$). In the case of the $HMXB_{s}$ shown in the sketch (\ref{fig:mspf}), the massive companion will continue evolving to explode as a supernova, forming a second neutron star. this system is called Double Neutron Stars. The first discovery of such system has curried by \citet{taylor1982new}. PSR J0737-3039 system (\citet{burgay2003increased}) is the first double neutron system to be observed as binary pulsars. with 22 ms period and an orbital period of order 2.4 hours, this makes it one of the best system to be used in testing the theory of general relativity and other theories of gravity. $HMXB_{s}$ can also form neutron star-black hole binary or even a black hole binary.\\
For the $LMXB_{s}$ case, the system evolves to be millisecond pulsar-white dwarf binary system.
For full review see (\citet{Lorimer2001})  
(\citet{backer1982millisecond})

\begin{figure}[H] 
\centering    
\includegraphics[width=1.0\textwidth]{MSformation.png}
\caption[Formation model of the Binary system]{Shows the formation and evolution of different types of binary systems, The image from \citep{Lorimer2001}  
}
\label{fig:mspf}
\end{figure}


\subsection{Magnetars}
Another type of pulsar with an ultra-magnetic field is the "Magnetars". They located in the upper right-hand of Figure (\ref{fig:pp}) shown by the green triangle, with magnetic field strengths of order $B \sim 10^{15}$ Gauss, which makes them one of the strongest cosmic magnetic field object in the universe. As suggested by \citet{thompson1993neutron}, the decay of magnetic field is considered to be the main source of magnetar energy, the magnetar model \footnote{\citet{duncan1992formation} for more details in different possibilities of the model}. They concluded: " Magnetic fields can deposit an enormous amount of energy outside a young neutron, and can catalyze the conversion of energy from neutrinos to electron pairs".\\
Depending on their observed emissions, two types of magnetars have been classified: Soft gamma repeaters (SGRs) which identified as high-energy transient sources. The second class is what called Anomalous X-ray pulsars (AXPs) which show continuing pulsation with rapidly spinning down in X-ray ranges, as well as some, have been observed as SGR-like bursts source. These two classes have spin period between 5 to 12 seconds and very short characteristic ages in order $ P / \dot{P} \sim 10^{3} - 10^{5}$ yr. (For full review see the book \citet{lewin2006compact}. Magnetars also have been observed as radio pulsar in the system \textbf{XTE J1810-197} which emits bright, narrow radio pulses with very high linear polarization (\citet{camilo2006transient}.%(\citet{israel2004accurate})


\subsection{Rotating Radio Transient (RRATs)}
Anew type of pulsars populations can be identified as the Rotating Radio Transients (RRATs), the yellow pox in the center to the right of Figure (\ref{fig:pp}). The discovery of the RRATs has been carried by \cite{mclaughlin2006transient} when eleven sources were identified as single pulses of radio emission using data from Parkes Multi-beam Pulsar Survey. The measure duration of these pulses range between 2 to 30 ms with flux density peak ranging from 0.1 to 3.6 Jy, while the average time interval between the burst found to be in order of 3 minutes to 3 hours, as well as the dispersion measure (DM) value which shows a constant value for all the eleven sources.



For full review see \citet{2010PhDT.......460K}







%********************************** %Second Section  *************************************

\section{Interstellar medium (ISM)} %Section - 1.2
\label{ISM}

%Add some more contents

The interstellar medium (ISM) is the environment between the star systems in a galaxy that contains ordinary matter, relativistic charged particles (cosmic rays) and magnetic field. The standard model consists of three components which are in pressure equilibrium. Two of these components were suggested by \citet{field1969cosmic} which are based on heating by low-energy cosmic rays, the cold dense phase $T < 300 K$ which consists of neutral and molecular hydrogen clouds and warm intercloud phase $T = 10^{4} K$ consisting ionized gas. The third phase is the hot ionize gas with temperature  $T = 10^{6} K$  which has been added by \citet{mckee1977theory}. their study showed that the hot ionized gas is a result of the supernova explosions which create a shock-waves that evaporate the cool clouds to hot medium.\\
As the radio emission from pulsars propagates through the ionized interstellar medium (\textbf{ISM}), it interacts with the free electrons leading to observable propagation effects. There are three main effects that affect the signals (pulses) from the pulsar traveling through the ISM: scintillation, scattering, and dispersion. To create a high precision timing, one needs to correct for these effects (e.g. \citet{armstrong1995electron})




\subsection{Dispersion}
\label{Dispersion}
As we mention in \cref{ISM}, the frequency dispersion is one of important characteristic of the radio signals from pulsars, when they propagate through the ionized interstellar medium. The refractive index of the gas in the ionization state can be obtained from the plasma frequency $\nu_{p}$ as,

\begin{equation}
\label{revractive index}
n = \left(1- \frac{\nu^{2}}{v^2} \right)^2
\end{equation}

and thus,the plasma frequency is \footnote{The electron density in ISM environment is given in $cm^{-3}$ which also can represent in kHz. As the approximate value of $n_e=0.03 cm^{-3}$ gives a plasma frequency of 1.5 kHz, then we can approximate the refractivity (n-1) to be $-2.4  \times10^{-10}$ for a 100 MHz. see the book for full text \cite{lyne2012pulsar} }  

\begin{equation}
\label{revractive index1}
\nu_p^2 = \frac{n_e e^{2}}{\pi m}
\end{equation}

where n is the refractive index of the ionized gas and $\nu$ is the frequency of the wave. e and m are the electronic charge and mass respectively. As the group velocity of the traveling pulses is $\nu_g = cn$, where c is the speed of light in the vacuum, thus for the given electron densities 

\begin{equation}
\label{group velocity}
\nu_g^2 = c \left(1 - \frac{n_e e^{2}}{2\pi m v^2} \right)
\end{equation}

The travel time T through the distance L, therefore, will be in the form 

\begin{equation}
\label{Travel Time}
T = \int_0^L \frac{dl}{v_g} = \frac{L}{c} + \frac{e^2 \int_0^L n_e dl}{2 \pi mcv^2}  = \frac{L}{c} + 1.345 \times 10^{-3} v^{-2} \int_0^L n_e dl  
\end{equation}

This equation shows the travel time of the free space (the first term) with an additional term which represents the dispersive delay t. the 
extra term is the what called the dispersion measure (DM)\footnote{The unit of the DM is a combination of the distance unit, parsecs = 3 $\times10^{18}$ and  the unit of the density, $cm^{-3}$ } given as

\begin{equation}
\label{Dispersion measure}
DM =  \int_0^L n_e dl
\end{equation}

Which measures the electron density between the pulsar and the telescope, with unit $ cm^{-3}$ pc.

From equation \ref{Travel Time} and \ref{Dispersion measure}, the delay due to dispersion can be written in the form 

\begin{equation}
\label{Dispersive Delay}
t =  \mathcal{D} \times \frac{DM}{\nu^2}
\end{equation}

Where $\mathcal{D}$ is the dispersion constant which given as 

\begin{equation}
\label{Dispersion Constant}
\mathcal{D}  = \frac{e^2}{4 \pi mc} = 4.1488 \times 10^{3} MHz^{2} pc^{-1
}cm^3
\end{equation}

If we have tow different frequencies ($\nu_{low}$ and $\nu_{high}$), then equation \ref{Dispersive Delay} can be written in another useful form as

\begin{equation}
\label{Dispersive Delay of two freq}
\Delta t =  \mathcal{D}.DM \times \left(\frac{1}{\nu^2_{low} } - \frac{1}{\nu^2_{high} } \right)
\end{equation}

The effect of this delay can be seen in the observations when the pulses with higher frequencies arrive earlier than that in low frequencies as shown in Figure (\ref{fig:DM effect}). The time differences of the received signal with bandwidth B (in MHz) can be calculated (in seconds) as

\begin{equation}
\label{Disp}
\Delta t =  8.3 \times 10^3 DM \nu^2 B 
\end{equation}


\begin{figure}[htbp!] 
\centering    
\includegraphics[width=0.8\textwidth]{pulsar_dm.png}
\caption[The frequency dispersion]{Shows the effect of the frequency dispersion which the pulses at high frequencies  arrive earlier for PSR B1641-45. The image from \citet{lyne2012pulsar}.  
}
\label{fig:DM effect}
\end{figure}

%The dispersion can be used to measure how far the pulsars are located from the earth. As the value of DM can be determined by measuring the distribution of the free electrons, \citet{2002astro.ph..7156C} have described a new model "NE2001" to approximate the distance of the pulsars.\\


\subsection{De-dispersion}
%Add section for the dm variation and another for structure function 
We have discussed in section \cref{Dispersion} the effects of the dispersion on the radio pulses source which causes the pulses at high frequencies arrive earlier than that at low frequencies. When the dispersive delay becomes larger than the period of the pulsars, the detection of a new pulsar will become more difficult. Therefore, we need to solve this effects. The method used to solve this effect is called the \textit{de-dispersion} and can be implemented by two way

\subsubsection*{Incoherent de-dispersion}
This method uses the technique of separating the received bandwidth into multiple sub-bands and then apply a suitable delay for each channel. This delay will adjust the channels to be aligned along the range of frequency and solve for the delay Figure (\ref{fig:de-DM}) (e.g \citet{large1971search}) 

\begin{figure}[htbp!] 
\centering    
\includegraphics[width=0.8\textwidth]{de_dm.png}
\caption[The de-dispersion]{Shows the broadened profile of the received pulses due to dispersion (top panel). The process of solving the dispersion delay by dividing sub-band into multiple channels and apply the suitable delay for each channel (lower panel).
Figer from Hand book pulsar astronomy by \citet{lorimer2005handbook} }
\label{fig:de-DM}
\end{figure}



\subsubsection*{Coherent de-dispersion}
This technique applies the processing of the full bandwidth signal before the detection, therefore, removing the effect of the interstellar dispersion with higher precession than incoherence techniques (e.g   
\citet{rickett1975radio}). This can be achieved by applying a phase delay, correspond to specific frequency, to the received signal in the filter. The filter then digitizes the signal before the detection by applying the Fourier transform techniques to a sequence of samples. Then, the phase in each discrete components of the spectrum can be delayed by appropriate values depending on the amount of the frequency. Finally, the reverse Fourier transform is used to return the signal the signal without the dispersion delay \citet{lyne2012pulsar}.      


\subsection{Scattering}
We outlined in section \cref{Dispersion} the dispersion effects of the radio pulses from pulsars when they pass through the ionized ISM, however, the inhomogeneities of the electron density along the line of sight causes scattering of the pulses. The mean effect of the inhomogeneities on the observed pulses causes broadening the pulses in time, therefore, a simple model of a thin screen was proposed by \citet{williamson1972pulse}. In this model, the scattered electrons along the line of sight from the pulsars to an observer leads to frequency dependence effect such as the pulse broadening Figure (\ref{fig:scut_model}).\\
As the wave propagate through an inhomogeneity, its phases change due to the refractive index and thus, by consider a screen with scale a in the midway between the pulsar and the observer, we can manifest this phase change $\Delta\theta$ by approximate the angle $\theta_0$ at the screen as

\begin{equation}
\label{angle}
\theta_{0} \approx \frac{\Delta\theta / k}{a} \approx \frac{e^2}{\pi m_e} \frac{\Delta n_e}{\sqrt{a}} \frac{\sqrt{D}}{f^2}
\end{equation}

where $k = (2\pi/c) \mu f$ and D is the distance to the pulsar\\
Similarly, an angular radius $\theta_d$ of the diffuse scatter disk around the source can be seen by an observer as

\begin{equation}
\label{anglular}
\theta_{d} = \frac{\theta_0}{2} \approx \frac{e^2}{2\pi m_e} \frac{\Delta n_e}{\sqrt{a}} \frac{\sqrt{D}}{f^2}
\end{equation}

the angular intensity distribution also observed to follow the gaussian probability distribution as 

\begin{equation}
\label{intinsity}
I(\theta)d\theta \approx \exp(-\theta^2/\theta^2_D) 2 \pi \theta d\theta
\end{equation}

The received scattered-waves therefore will travel an additional distance, due to the small change in the direction (see Figure \ref{fig:scut_model}), and thus the path length will also increase which eventually leads to a geometrical time delay $\Delta t(\theta)$ as

\begin{equation}
\label{g delay}
\Delta t(\theta) = \frac{\theta^2 D}{c}
\end{equation}

This effect can be used to obtain the observed intensity, I, as a function of time

\begin{equation}
\label{obs-intinsity}
I(t) \approx \exp(-c \Delta t/(\theta^2_d D)) \equiv e^{-\Delta t/\tau_s}
\end{equation}

Where

\begin{equation}
\label{tau}
\tau_s = \frac{\theta^2_d D}{c} = \frac{e^4}{4 \pi^2 m^2_e} \frac{\Delta n^2_e}{a} d^2 f^{-4}
\end{equation}
This what shows the observed tail with the exponential shape that appear on the received pulse signal. 


\begin{figure}[H] 
\centering    
\includegraphics[width=0.85\textwidth]{PSRs_thin_screen.png}
\caption[The de-dispersion]{Shows the broadened pulses due to screen scattering model.
Figure from Hand book pulsar astronomy by \citet{lorimer2005handbook} }
\label{fig:scut_model}
\end{figure}


\subsection{Scintillation}
Beside the dispersion and scattering of the ISM, an additional effect called "interstellar Scintillation" also can be observed. The scintillation is defined as a short-term intensity variation appear on the pulsar observations. This caused by the random electron density between the pulsar and the observer.\\
Using a similar model of the scattering, the thin screen model can be used for analysis
of the pulsars scintillation. Figure (\ref{fig:sscint_model}) shows that the scattered radiation from the pulsar to an observer presented in a thin screen leads to random irregularities in the refractive index which causes phase variations of the wavefront. This phase differences then will be received along the line of sight by the observer as a scintillation pattern.\\
The scintillation bandwidth $B_{s}$ can be given by the simple formula

\begin{equation}
\label{dm scintillation}
B_{s} \approx \frac{8 \pi^2 a c}{D^2 (\Delta n_e)^2 \lambda^4}
\end{equation}

Where D is the distance between the pulsar and the observer, a is the length of the scattered ray path, c is the speed of light and $\lambda$ is the wavelength (\citet{lyne2012pulsar}).

\begin{figure}[H] 
\centering    
\includegraphics[width=0.85\textwidth]{scintt.png}
\caption[scintilation]{Shows the random irregularities of refractive index as represented by thin-screen model of scintillation}
\label{fig:sscint_model}
\end{figure}


%********************************** % Third Section  *************************************
\section{Pulsar timing}  %Section - 1.3 
\label{section1.3}
%#####Add more about in pulsar timing. why do we do pulsar timing?
Pulsar timing is a technique used to study the spin evolution of the pulsars by measuring the time of arrival (TOAs) of the transmitted pulses from pulsars to the Earth. Study and analyze TOAs allow us to deliver research and knowledge in many areas. first, studying the origins of the pulsars and construct the properties associated with it (e.g period, spin and spin down, age, etc), second, using pulsars as a tool to measure the positions of the source with high accuracy values, third, exploring the propagation of the pulses through the ISM; therefore, many effects of ISM can be studied (e.g dispersions of the pulse, see \ref{Dispersion}).\\
Similarly, apply pulsar timing on the pulsars in a binary system enable to measure their orbits and rotations slowdown and testing the Einstein's theory of Relativity. One of the main applications of pulsar timing also, to detect low-frequency gravitational waves (e.g stochastic background). In this section, we give overview of pulsar timing techniques, TOAs and timing model and its parameters.\\
this section is based on \citet{lorimer2005handbook} and the recent review from \citet{manchester2017pulsar}\\


%Pulsars are rotating neutron stars that emit beams of radiation in the different spectrum (e.g radio, optical, X-ray and $\gamma$-ray). , time of the writing, with periods (Ps) range between 1 ms to 15 s. Different categories of pulsars vary on their parameters. Milliseconds \Cref{MSP}, for instance, have short Ps, 1 ms to 10 ms and very small period derivative or the so-called "slowdown rate", ($\dot{P}$). On the other hand, Normal pulsars \cref{Normal Pulsars} have long Ps, 3 s to 0.3 s. ===========

%########## its repeated
%######### Now there is a pulsar with period of 33.05s. new paper but not published yet

\subsection{Time of arrivals (TOAs)}
The time of arrival (TOA) is defined as "the arrival time of the nearest pulse to the mid-point of the observation" \citet{lorimer2005handbook}. In pulsar timing, the TOAs taken from pulsars observations over long interval enable us to create a timing model; which then can be used to determine the parameters of the pulsars, with a good accuracy, and perform the analyses of their evolution.\\
Practically, to measure the arrival times of the pulses, the pulsars data need to be folded at the period of the pulsars. the result of this process is the so-called "average pulse profile" which used to establish the timing.

\subsection{Template matching}
\label{cross-correlation}
The generated average pulse profile is very stable and therefore, the TOAs can be determined with high accuracy. The cross-correlation method is considered the best methods for measuring the timing of arrival of pulsars. In this methods, the observed profile is matched with high signal to noise (S/N) template. these templates are constructed from a set of early observations at a known range of frequency.\\
Suppose the average pulse profile is given by $\mathcal{P}(t)$, $\mathcal{T}(t)$ for the template and noise $\mathcal{N}(t)$ thus

\begin{equation}
\label{cross-cor}
\mathcal{P}(t) =  a + b \mathcal{T}(t - \tau) + \mathcal{N}(t)  
\end{equation}

where a is the arbitrary offset and b is scaling factor and $\tau$ is the phase offset which shows the time-shifted between the template and the profile; then by applying the cross-correlate technique, TOAs can be measured.\\
TOAs can be measured with high precisions; however, many effects, either associated with the pulsars or as systematic effects, can limit the measurements of pulsars TOAs. This introduces an uncertainty in the measurements of TOAs which can be calculated by the following formula

\begin{equation}
\label{toa precision}
\sigma_{TOA} \simeq  \frac{S_{sys}}{\sqrt[]{t_{obs} \Delta f}}  \times \frac{P\delta^{3/2}}{S_{mean}}
\end{equation}

Where $S_{sys}$ is the flux density of the system, $\Delta f$ is the observed bandwidth, P is the pulse period, $\delta = W/P$ is the pulse duty cycle and the mean flux density is given by $S_{mean}$.





\subsection{Pulsar timing-model parameters}
%check the range approx of the values for each parameters
In the pulsar timing, the measured TOAs from the received pulses at the observatory needs to be fitted to a model by using an appropriate method (e.g cross-correlation Section \ref{cross-correlation}). Considering this fitting, there will be variations between the TOAs at the telescope and the time of emission at the pulsar; hence, a timing model is required to correct all the effects which limit our ability to measure the average TOAs with high accuracy.\\
Here we show some of the parameters of the timing model and how they appear in the timing residuals of the observations (Figure \ref{fig:timing model}) . For full review see \citet{edwards2006tempo2}


\begin{figure}[H] 
\centering    
\includegraphics[width=0.85\textwidth]{PSRs_timing_example.png}
\caption[scintilation]{Shows timing-model parameters and their effects on the residual structure. Panel (a) shows an accurate timing model which all the residual centered around 0 ms. Panel (b) The spin down (or frequency derivative) errors which give a quadratic increase shape. Panel (c) The position error which gives a sinusoidal shape due to the variation with the period of the Earth around the sun (1 year). Panel (d) shows inaccurate pulsar's proper motion measurements which result in 1-year sinusoid shape with quadratically-increasing amplitude. \citet{lorimer2005handbook}}
\label{fig:timing model}
\end{figure}


\subsection*{Barycentric corrections}
Observatories on Earth measure the pulse TOAs by an atomic time standard called "Terrestrial Time (TT)"; Measuring the TOAs using the observatory clock is known "topocentric arrival time" which occurs in non-inertial frame, due to rotating Earth orbiting the Sun; therefore, we need to transfer this to an inertial reference frame which represents the center of mass for the solar system. The frame of Solar system barycenter "SSB" is approximated as a perfect inertial frame to measure the TOAs "barycentric arrival time".\\
The transformation from the topocentric TOA to barycentric TOAs is given by 

\begin{equation}
\label{SSB eq}
t_{SSB} = t_{topo} + t_{corr} -  \Delta D/f^2 +  \Delta_{R_{\odot}} +  \Delta_{S_{\odot}} +  \Delta_{E_{\odot}} 
\end{equation}

We can use equation (\ref{SSB eq}) as a reference to explain each term as the following

\subsubsection*{clock corrections}
The term $t_{corr}$ shows the corrections of the observatory clock time. this regarding the correction of the observatory time to Coordinated Universal Time (UTC) and terrestrial time; (details about time variable on different reference frames see \citet{mccarthy2004iers}).

\subsubsection*{frequency corrections}
$\Delta D/f^2 $ is represent the dispersion measure and dispersion constant corrections. As we showed in Section \Cref{Dispersion} the pulses through the ISM are delayed due to DM; which shows that the TOAs depend on the observed frequency (f).   


\subsubsection*{Romer delay}
 The term $\Delta_{R_{\odot}}$ shows the so-called R$\ddot{o}$mer. This is the vacuum delay between the arrival of the pulse at the observatory and SSB frame.  

\begin{equation}
\label{romer}
\Delta_{R_{\odot}} = - \frac{\bf{\overrightarrow r}.  \hat{\bf s}}{\bf c}
\end{equation}

  Where $\bf{\overrightarrow r}$ is a vector pointing from SSB frame toward the observatory. and more precisely, can be divided to two components, $\bf{\overrightarrow r_{SSB}}$ which connect the SSB with the center of the Earth (geo-centre) and $\bf{\overrightarrow r_{EO}}$ which connect the geo-centre with the phase center of the telescope. the second vector $\hat{\bf s}$ in equation (\ref{romer}) is pointing from the SSB frame to the position of the pulsar.  


\subsubsection*{Shapiro delay}
The Shapiro delay  $\Delta_{S_{\odot}}$
is a time delay of the pulses due to the curvature of space-time. The total of this delay can be measured by adding all the masses in the solar system as 

\begin{equation}
\Delta_{S_{\odot}} = -2 \sum_{i} \frac{GM_i}{c^3} ln \left[ \frac{ \hat {\bf s}.\bf{\overrightarrow r^E_i}  + r^E_i}{  \hat {\bf s}.\bf{\overrightarrow r^P_i}  + r^P_i } \right]
\end{equation}

where G is Newton gravitational constant, $M_i$ is the mass of the included body $i$, $\bf{\overrightarrow r^P_i}$ and $\bf{\overrightarrow r^E_i}$ are pulsar position and telescope position relative to the body $i$ respectively.
 
 
\subsubsection*{Einstein delay}
The Einstein delay is the combined effect of time dilation due to both; the Earth motion (change the distance between the observatory and the pulsar, or the orbital period) and the gravitational redshift caused by the masses of the bodies in the solar system. Is given as 
%This delay depends on the orbital phase in an elliptical orbit. depending on the mass of both the pulsar and companion, the time  

\begin{equation}
\Delta_E = \gamma sin E,
\end{equation}

which shows a sinusoid shape in the residual, then, the amplitude $\gamma $ is given as

\begin{equation}
\gamma = T^{2/3}_{\odot} \left(\frac{P_b}{2\pi} \right)^{1/3} e \frac{m_2(m_1 + 2m_2)}{(m_1+m_2)^{4/3}}
\end{equation}
where $e$ is the eccentricity.

Note that


\subsection{Pulsar timing arrays (PTAs)}
Pulsar timing array (PTA) is a set of arrays to perform high-precision timing for pulsar observations over a long time duration. The aims of this project are, detect the low-frequency gravitational waves, enhance the Solar System ephemeris and search for irregularities in the terrestrial time standards.
Pulsar timing can be obtained by searching for correlation in the signals of the pulsar timing residuals (see Figure \ref{fig:pulsar timing}). \citet{hobbs2009pulsar}\\

\begin{figure}[htbp!] 
\centering    
\includegraphics[width=0.85\textwidth]{pt11.png}
\caption[Pulsar timing]{Artist's conception shows the correlation of the signal from set of pulsars. Pulsar timing is using the residuals from each signals to obtains the TOAs which then can be used to detect the gravitational waves background. Image credit: David J. Champion }
\label{fig:pulsar timing}
\end{figure}


The project consists three individual projects as the following: 
\subsubsection*{European Pulsar Array (EPTA)}
This project consists of Jodrell Bank, Effelsberg, Westerbork and Nancay telescopes each 100-m diameter in Europe. Using these telescopes, the project is producing a high precision timing observation for more than 15 pulsars.

\subsubsection*{Parkes Pulsar Array (PPTA)}
Making use of Parkes radio telescope, 64-m in Australia, to observe 20 pulsars for one hour every two-three weeks. The telescope uses two receivers, 10cm-50cm and 20cm to perform the observation of one hour successively.

\subsubsection*{North American Nanohertz Observatory for Gravitational Waves (NANOGrav))}
Using Arecibo radio telescope, 300-m, and Green Bank telescope in the United States to carry out monthly observations for 24 pulsars.\\


\subsubsection*{International Pulsar Timing Array (IPTA)}
As the main aim of PTAs is to deliver a direct detection of low-frequency gravitational waves, the data from the three PTAs have combined in the so-called International Pulsar Timing Array (IPTA). see \citet{hobbs2010international}


































